% chktex-file 3

\documentclass[11pt, letterpaper, twoside]{article}
\usepackage[utf8]{inputenc}
\usepackage[fleqn]{amsmath}
\usepackage[british]{babel}
\usepackage[T1]{fontenc}
\usepackage[inner=2.5cm, outer=2.5cm, top=3cm, bottom=3cm]{geometry}
\usepackage[hidelinks]{hyperref}
\usepackage{cancel}
\usepackage{cleveref}
\usepackage{aligned-overset}
\usepackage{booktabs}
\usepackage{changepage}
\usepackage[acronym,toc,nogroupskip,nopostdot,seeautonumberlist,nonumberlist]{glossaries}

\usepackage{pgfplots}
\pgfplotsset{compat=newest}
\usetikzlibrary{plotmarks}
\usetikzlibrary{arrows.meta}
\usepgfplotslibrary{patchplots}
\usepackage{grffile}

\bibliographystyle{acm}

% \allowdisplaybreaks%

\title{SLAM}
\author{Martino Pilia}
\date{\today}

\newglossary[symg]{symbol}{syms}{symo}{Symbols}

\setglossarysection{section}
\makeglossaries%

\newacronym
    {vio}
    {VIO}
    {visual-inertial odometry}

\newacronym
    {slam}
    {SLAM}
    {simultaneous localisation and mapping}

\newacronym
    {sota}
    {SotA}
    {state-of-the-art}

\newacronym
    {ros}
    {ROS}
    {Robot Operating System}

\newacronym
    {rpc}
    {RPC}
    {remote procedure call}

\newacronym
    {lts}
    {LTS}
    {long-term support}

\newacronym
    {bow}
    {BoW}
    {bag-of-words}

\newacronym
    {ba}
    {BA}
    {bundle adjustment}

\newacronym
    {pnp}
    {PnP}
    {perspective-n-point}

\newacronym
    {spark}
    {SPARKlab}
    {Sensing, Perception, Autonomy, and Robot Kinetics laboratory}

\newacronym
    {rtabmap}
    {RTAB-Map}
    {Real-Time Appearance-Based Mapping}

\newacronym
    {introlab}
    {IntRoLab}
    {Intelligent Robot Lab}

\newacronym
    {rgbd}
    {RGB-D}
    {colour-depth}

\newacronym
    {imu}
    {IMU}
    {inertial measurement unit}

\newacronym
    {ir}
    {IR}
    {infrared}

\newacronym
    {sgm}
    {SGM}
    {semi-global matching}

\newacronym
    {fov}
    {FoV}
    {field of view}

\newacronym[
    plural=DoF,
    firstplural=degrees of freedom
    ]
    {dof}
    {DoF}
    {degree of freedom}

\newacronym
    {cde}
    {CDE}
    {code, data, and environment}

% \newacronym
%     {}
%     {}
%     {}


\begin{document}

\maketitle

\begin{abstract}
    This document is a brief report of activities performed between November 2019
    and January 2020, concerning the investigation of \gls{sota} tools
    for \gls{slam} based on \gls{VIO}, meant to identify a suitable starting point
    to implement indoor localisation solutions on mobile devices.
\end{abstract}

\tableofcontents

\glsfindwidesttoplevelname[\acronymtype]
\printglossary[type=\acronymtype,style=alttree,title=Abbreviations,nonumberlist]

\newpage

\section{Introduction}

\subsection{Scope}

This document presents a preliminary evaluation of mapping and localisation
tools, aimed to explore the current \gls{sota} and to identify a suitable
starting point for further development of mobile localisation and mapping
applications.

\subsection{OpenVSLAM}

OpenVSLAM\cite{openvslam2019} is a \gls{slam} system developed at the National
Institute of Advanced Science and Technology in Japan. It consists of a C++
library based on OpenCV that allows to perform \gls{slam} based on pure visual
odometry, and supports multiple camera configurations as input, including
monocular, stereo, and RGB-D, with several lens models (pinhole, fisheye,
equirectangular). While being very young, the project has good quality and
coding standards, and it is currently under active development.

OpenVSLAM provides fairly good visual odometry based on point matching, by
extracting ORB corners\cite{rublee2011orb} in each frame. It implements loop
closure detection, based on a \gls{bow} model: loop closure candidates are
searched among frames containing a sufficiently high number of corresponding
visual words and are confirmed by geometric verification. If a suitable
candidate is found, a loop closure edge is added to the pose graph, that is in
turn optimised with g2o\cite{grisetti2011g2o}. This mechanism works reasonably
in some environments, but it easily fails in presence of perceptual aliasing
(e.g.\ when navigating through similar corridors in an office space), due to
the lack of any mechanism to cope with outliers in the loop closure detection.

Localisation is also based on \gls{bow}, by retrieving a frame with similar
visual words and subsequently recovering the camera pose through \gls{pnp}.
This approach works on a single session \gls{slam}, but it is likely not robust
enough to handle multiple sessions with varying conditions (illumination,
crowding, etc.) or image data from different sensors.

While OpenVSLAM is reasonably (but not fully) optimised for speed, it is not
optimised for memory usage. The map can be stored to disk in messagepack
format, to be subsequently loaded in a new session, and it is kept in memory in
a JSON structure, as most other data structures in the software, therefore
incurring in a serious memory overhead. This is a limiting factor, making it
impossible to map more than a couple floors of a building such as Kampusareena
using a 32~GB laptop. Another limiting factor is the time required to perform
\gls{ba}, that grows without bounds with the size of the pose graph, requiring
several minutes to complete a loop closure on a map with a few thousand
keyframes.

\subsection{The Robot Operating System}

\gls{ros}\footnote{\url{https://www.ros.org/}} is a software framework
developed at Stanford University as a platform for robotic research. \gls{ros}
has a modular design and allows to break a system into multiple, re-usable
applications (e.g.\ controlling different sensors or peripherals). Applications
can be distributed as \textit{packages} and are executed as \textit{nodes},
communicating over the network so they can effectively run over different
computers. \gls{ros} nodes communicate by sending network packages through
streams called \textit{topics}, and through an \gls{rpc} call mechanism (known
as \gls{ros} \textit{services}).

\gls{ros} 1.x is released in \textit{distributions}, each of those is supported
on specific versions of Ubuntu. \gls{ros} 2.x is the new generation system,
which is largely incompatible with 1.x and it is meant to slowly replace it.

A detailed introduction to the \gls{ros} is beyond the scope of the present
document, and comprehensive documentation and tutorials are available on the
official wiki.\footnote{\url{http://wiki.ros.org/}} If not familiar with
\gls{ros}, a walk through the tutorials is highly
recommended.\footnote{\url{http://wiki.ros.org/ROS/Tutorials}}

\bibliography{bibliography}

\end{document}
